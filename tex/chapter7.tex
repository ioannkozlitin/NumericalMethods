\documentclass{report}
\usepackage{amsmath}
\usepackage{amsfonts}
\usepackage{graphicx}
\usepackage[T2A]{fontenc}     
\usepackage[russian]{babel} 
\usepackage{titlesec}

\newcommand{\seminarTitle}[1]{
    \begin{center}
        {\LARGE \bfseries #1}
    \end{center}
    \vspace{1em} 
}

\titleformat{\section}[block]{\filcenter\bfseries\large}{\thesection}{1em}{}
\titleformat{\subsection}[block]{\filcenter\bfseries\normalsize}{\thesubsection}{1em}{}

\begin{document}

\seminarTitle{Теория к семинару №7}

\section*{Теоретический материал}
Одномерное уравнение переноса имеет следующий вид:
\begin{equation} \label{c7eq1}
    \begin{cases}
        \displaystyle \frac{\partial u}{\partial t} + c \frac{\partial u}{\partial x} = f(x,t) \\
        u(0,t) = \mu_1(t) \\
        u(x,0) = \mu_2(x)
    \end{cases}.
\end{equation}
Здесь $c > 0$ - скорость переноса, $\mu_1(t)$ - граничное, а $\mu_2(x)$ - начальное условия. Аппроксимируя производные конечными разностями, получим ряд разностных схем для этого уравнения:
\begin{align}
    \frac{\hat{u}_n - u_n}{\tau} + c \frac{u_n - u_{n-1}}{h} = \Phi_n, \label{c7eq2}
\end{align}
\begin{align}
    \frac{\hat{u}_{n-1} - u_{n-1}}{\tau} + c \frac{\hat{u}_n - \hat{u}_{n-1}}{h} = \Phi_n, \label{c7eq3}
\end{align}
\begin{align}
    \frac{\hat{u}_n - u_n}{\tau} +  c \frac{\hat{u}_n - \hat{u}_{n-1}}{h} = \Phi_n, \label{c7eq4}
\end{align}
\begin{align}
    \frac{\hat{u}_n + \hat{u}_{n-1} - u_n - u_{n-1}}{\tau} + c \frac{\hat{u}_n + u_n - \hat{u}_{n-1} - u_{n-1}}{h} = 2 \Phi_n, \label{c7eq5}
\end{align}
где $\tau$ - шаг по времени сетки, $h$ - шаг по пространству, а $\Phi_n = f(x_{n-1/2}, t_{n+1/2})$ - значение функции из правой части уравнения в середине ячейки.

Первые две схемы условно устойчивы, две последние – безусловно устойчивы. Граница устойчивости задается с помощью неравенства, в которое входит число Куранта $\kappa = c\tau/h$. Первая схема устойчива при $\kappa \leq 1$, а вторая – при $\kappa \geq 1$. Таким образом, их условия устойчивости противоположны. Это дает возможность построить из них так называемую составную схему, называемую также схемой Карсона. Идея в том, чтобы при $\kappa \leq 1$ использовать первую схему, а при $\kappa \geq 1$ - вторую. В итоге составная схема получается безусловно устойчивой. Если число Куранта $\kappa$ переходит через единицу, например из-за изменения скорости переноса или неравномерности сетки, то часть шагов может быть сделана по первой схеме, а часть - по второй. Оказывается, что составная схема оказывается несколько точнее безусловной устойчивой чисто неявной схемы (3 схема в списке).

Первая схема - явная, все остальные - формально неявные, хотя на самом деле считать по ним не труднее, чем по явным, поскольку все неизвестные величины на следующем временном слое получаются либо из граничного условия, либо из начального, либо из результата предыдущего расчета. Важно лишь соблюдать правильный порядок вычислений - от левой границы области расчета к правой и от более раннего временного слоя к более позднему.


\section*{Задание к семинару №7}
Задачу будем решать на отрезке  $[0; 100]$ по пространству и   [0; 1] по времени. Шаг равномерной сетки по пространству $h = 0.1$, шаг по времени $\tau = 0.01$. Скорость переноса $c = 50$. Правая часть - нулевая. Начальное условие
\begin{equation} 
    \mu_2(x) = \frac{1}{1 + \left( \displaystyle \frac{x-20}{10} \right)^{10}}. \label{c7eq6}
\end{equation}
Граничное условие $\mu_1(t) = \mu_2(-ct)$.

Реализовать составную схему, чисто неявную схему (3 схема) и схему с полусуммой (4 схема). Положить на один график точное решение задачи $\mu_2(x-ct)$, а также результаты расчета по всем трем схемам. Объяснить поведение кривых численного решения.


\section*{Приложение}
Для реализации и решения задачи одномерного уравнения переноса в Python можно использовать следующие инструменты и библиотеки:

\subsection*{NumPy}
\texttt{NumPy} — это фундаментальная библиотека для научных вычислений в Python. Она поддерживает большие многомерные массивы и матрицы, а также предоставляет множество высокоуровневых математических функций для их обработки. В контексте данной задачи \texttt{NumPy} можно использовать для:
\begin{itemize}
    \item Создания сеток по пространству и времени с использованием функций \texttt{numpy.linspace()} или \texttt{numpy.arange()}.
    \item Выполнения векторизованных операций при реализации разностных схем.
\end{itemize}

\subsection*{Matplotlib}
\texttt{Matplotlib} — это библиотека для создания статических, анимированных и интерактивных визуализаций в Python. Она будет полезна для:
\begin{itemize}
    \item Построения графиков численных решений и их сравнения с точным решением уравнения.
    \item Визуализации эволюции решения во времени и анализа поведения различных разностных схем с помощью функций \texttt{matplotlib.pyplot.plot()} и \texttt{matplotlib.pyplot.imshow()}.
\end{itemize}


\end{document}
