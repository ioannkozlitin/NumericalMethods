\documentclass{report}
\usepackage{amsmath}
\usepackage{amsfonts}
\usepackage{graphicx}
\usepackage[T2A]{fontenc}      
\usepackage[russian]{babel}
\usepackage{listings} 
\usepackage{titlesec}

\titleformat{\section}[block]{\filcenter\bfseries\large}{\thesection}{1em}{}
\titleformat{\subsection}[block]{\filcenter\bfseries\normalsize}{\thesubsection}{1em}{}

\begin{document}

\section*{Задание к семинару №8}
Решить квазилинейное уравнение переноса
\begin{equation} \label{c8eq1}
	\begin{cases}
		\displaystyle \frac{\partial u}{\partial t} + u \frac{\partial u}{\partial x} = 0 \\
		\displaystyle u(x,0) = \frac{1}{1 + \left( \displaystyle \frac{x-50}{10} \right)^4} \\
		u(0,t) = u(0,0) 
	\end{cases}.
\end{equation}
Задачу следует решать на отрезке $[0; 100]$ по пространству. Шаг равномерной сетки по пространству $h = 1$, шаг по времени $\tau = 0.05$. Число шагов по времени $N = 1000$.

Расчет вести с помощью консервативной чисто неявной схемы для квазилинейного уравнения переноса
\begin{equation} \label{c8eq2}
	\frac{\hat{u}_n - u_n}{\tau} + \frac{\hat{u}^2_n - \hat{u}^2_{n-1}}{2h} = 0.	
\end{equation}

\subsection*{Приложение}
Для рисования графиков и создания анимации в Python можно использовать библиотеку \texttt{matplotlib}. Для каждого временного шага необходимо вызывать функцию \texttt{plot} для построения графика решения на текущем временном слое.
Для сохранения анимации можно использовать функцию ani.save('animation.mp4', writer='ffmpeg', fps=30)

Пример кода на Python для построения анимации:
\begin{lstlisting}[language=Python]
    %matplotlib notebook
    import numpy as np
    import matplotlib.pyplot as plt
    import matplotlib.animation as animation
    
    fig, ax = plt.subplots()
    
    ax.set_xlim(-1.5, 1.5)
    ax.set_ylim(-1.5, 1.5)
    
    line, = ax.plot([], [], lw=2)
    
    def init():
        line.set_data([], [])
        return line,
    
    def update(frame):
        theta = np.radians(frame) 
        x = np.cos(theta) 
        y = np.sin(theta)
        line.set_data([0, x], [0, y]) 
        return line,
    
    ani = animation.FuncAnimation(fig, update, 
                                  frames=np.arange(0, 361, 6), 
                                  init_func=init, blit=True)
    
    plt.show()
    
    
\end{lstlisting}
\end{document}
