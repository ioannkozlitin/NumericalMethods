\documentclass{report}
\usepackage{amsmath}
\usepackage{amsfonts}
\usepackage{graphicx}
\usepackage[T2A]{fontenc}      
\usepackage[russian]{babel}
\usepackage{listings} 
\usepackage{titlesec}

\titleformat{\section}[block]{\filcenter\bfseries\large}{\thesection}{1em}{}
\titleformat{\subsection}[block]{\filcenter\bfseries\normalsize}{\thesubsection}{1em}{}

\begin{document}


\section*{Задание к семинару №10}
Решить двумерное однородное уравнение теплопроводности с граничными условиями Дирихле 
\begin{equation} \label{c10eq1}
	\begin{cases}
		\displaystyle \frac{\partial u}{\partial t} = k \left( \frac{\partial^2 u}{\partial x^2} + \frac{\partial^2 u}{\partial y^2} \right) \\
		\displaystyle u(x,y,0) = \mu(x,y) \\
		u(0,y,t) = \mu(0,y), \qquad u(a,y,t) = \mu(a,y)\\  
		u(x,0,t) = \mu(x,0), \qquad u(x,b,t) = \mu(x,b)
	\end{cases}
\end{equation}
в области $(x,y,t) \in [0; a] \times [0; b] \times [0; T]$.

Функция $\mu(x,y)$ задается следующим образом ($\texttt{\lstinline|x|}$ и $\texttt{\lstinline|y|}$ - вектора, содержащие координаты узлов прямоугольной сетки):
\begin{lstlisting}[language=Python, basicstyle=\small]
import numpy as np

def mu(x, y):
	z = np.zeros((len(y), len(x)))
	for i in range(len(x)):
	  for j in range(len(y)):
	    z[j, i] = -0.01 * np.sin(x[i]) + 0.05 * np.sin(y[j])
	return z
	\end{lstlisting}
Параметры области: $a = 6\pi$, $b = 4\pi$, $T = 10$. Выбрать равномерную сетку с $h_x = h_y = \pi/30$ и шагом по времени $\tau = 0.1$. Коэффициент теплопроводности $k = 0.2$.

\section*{Указания}
Для решения использовать эволюционно-факторизованную схему
\begin{equation} \label{c10eq2}
    \begin{cases}
        \begin{aligned}
            &\left(E - \frac{\tau}{2} \Lambda_x \right) \nu = \Lambda_x u + \Lambda_y u \\
            &\left(E - \frac{\tau}{2} \Lambda_y \right) \Delta u = \nu \\
            &\hat{u} = u + \tau \Delta u
        \end{aligned}
    \end{cases}.
\end{equation}


Здесь $\Lambda_x$ и $\Lambda_y$ - операторы пространственного дифференцирования. Проблем с промежуточным граничным условием здесь не возникает, так как $u$ на границе не меняется со временем, а значит на границе $\hat{u} - u = 0$ и $\nu = 0$.

Операторы $P_x = E - \frac{\tau}{2} \Lambda_x$ и $P_y = E - \frac{\tau}{2} \Lambda_y$ могут быть записаны в матричной форме с помощью замены $\Lambda_x$ и $\Lambda_y$ на соответствующие матрицы пространственного дифференцирования $L_x$ и $L_y$. При этом надо помнить, что
\begin{equation} \label{c10eq3}
	\begin{split}
		\Lambda_x u = u L_x \\
		\Lambda_y u = L_y u
	\end{split},
\end{equation}
т.е. матрицы пространственного дифференцирования $L_x$ и $L_y$ умножаются на матрицу значений в узлах сетки $u$ с  разных сторон. Аналогично при обращении операторов $P_x$ и $P_y$ надо использовать правое и левое матричное деление соответственно.

Не следует применять операторы пространственного дифференцирования $\Lambda_x$ и $\Lambda_y$ к крайним строкам и столбцам матрицы $u$, так как в таком случае в $\Delta u$ в крайних строках и столбцах будут содержаться ненулевые значения, что приведет к нарушению граничных условий в расчете.

\section*{Примечание}

Отображать двумерное решение на каждом временном слое можно с помощью функции $\texttt{\lstinline|mesh|}$.

\begin{lstlisting}[language=Python]
import numpy as np
import matplotlib.pyplot as plt
from mpl_toolkits.mplot3d import Axes3D

x = np.linspace(0, 6*np.pi, 100)
y = np.linspace(0, 4*np.pi, 100)
x, y = np.meshgrid(x, y)
z = -0.01 * np.sin(x) + 0.05 * np.sin(y)

fig = plt.figure()
ax = fig.add_subplot(111, projection='3d')

ax.plot_surface(x, y, z, cmap='viridis')

ax.set_xlabel('X')
ax.set_ylabel('Y')
ax.set_zlabel('U')

plt.show()

\end{lstlisting}

\end{document}
