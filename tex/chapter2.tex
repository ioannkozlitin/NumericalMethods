\chapter{}

\section{Теоретический материал к семинару №2}

Схема Рунге-Кутты для решения задачи Коши
\begin{equation} \label{c2eq1}
	\begin{cases}
		\displaystyle \frac{d\mathbf{u}}{dt} = \mathbf{f} \left( \mathbf{u}, t \right) \\
		\mathbf{u}(t_0) = \mathbf{u}_0
	\end{cases}
\end{equation}
имеет следующий вид:
\begin{equation} \label{c2eq2}
	\begin{split}
		&\mathbf{u}_{n+1} = \mathbf{u}_n + \tau_n \sum_{k=1}^{s} b_k \boldsymbol{\omega}_k, \tau_n = t_{n+1} - t_n; \\
		&\boldsymbol{\omega}_k = \mathbf{f} \left( \mathbf{u}_n + \tau_n \sum_{l=1}^{L} \alpha_{kl} \boldsymbol{\omega}_l, t_n + \tau_n a_k \right), 1 \le k \le s.
	\end{split}
\end{equation}
Здесь $\tau_n$ - шаги по времени, $s$ - число стадий, коэффициенты $\alpha_{kl}$ образуют матрицу Бутчера $\mathbf{A}$, a $a_k$ и $b_k$ - элементы векторов $\mathbf{a}$ и $\mathbf{b}$, вместе с матрицой Бутчера полностью задающих схему Рунге-Кутта.

Для реализации на компьютере с использованием Python удобнее записать \eqref{c2eq2} в векторной форме
\begin{equation} \label{c2eq3}
	\begin{split}
		&\mathbf{u}_{n+1} = \mathbf{u}_n + \tau_n \boldsymbol{\omega} \mathbf{b}^T, \tau_n = t_{n+1} - t_n; \\
		&\boldsymbol{\omega}_k = \mathbf{f} \left( \mathbf{u}_n + \tau_n \boldsymbol{\omega} \boldsymbol{\alpha}^T_k, t_n + \tau_n a_k \right), 1 \le k \le s,
	\end{split} 
\end{equation}
где $\boldsymbol{\omega}_k$ - $k$-тый столбец матрицы промежуточных результатов $\boldsymbol{\omega}$, первоначально полагаемой нулевой, $\mathbf{b}$ - вектор-строка коэффициентов $b$ и $\boldsymbol{\alpha}_k$ - $k$-тая строка матрицы Бутчера. Верхний индекс $T$ означает транспонирование.

\centering{\section*{Задачи к семинару №2}}
\begin{enumerate}
\item Записать расчетные формулы для схемы Кутта, если
\begin{equation} \label{c2eq4}
	\mathbf{A} = 
		\begin{pmatrix}
		0 & 0 & 0 & 0 \\
		1/2 & 0 & 0 & 0 \\
		0 & 1/2 & 0 & 0 \\
		0 & 0 & 1 & 0 \\
		\end{pmatrix},
	\qquad
	\mathbf{a} = 
		\begin{pmatrix}
		0 \\
		1/2 \\
		1/2 \\
		1 \\
		\end{pmatrix},
	\qquad
	\mathbf{b} = 
		\begin{pmatrix}
		1/6 \\
		1/3 \\
		1/3 \\
		1/6 \\
		\end{pmatrix}^T.
\end{equation}
\item Перейти к длине дуги в задаче
\begin{equation} \label{c2eq5}
	\begin{cases}
		\displaystyle \frac{du}{dt} = u^2 + t^2 \\
		u(t_0) = u_0
	\end{cases}.
\end{equation}
\item Реализуйте схему Кутта на компьютере, используя Python (или MATLAB) и соответствующие библиотеки.\\
Тестовые функции (правые части):

a)  

Python:
\begin{minted}[linenos=false,frame=lines,framesep=2mm,baselinestretch=1.2,fontsize=\footnotesize]{python}
def myfun(t, u):
    return u + t**2 + 1
\end{minted}

MATLAB:
\begin{matlablisting}
	\begin{Verbatim}
function y = f(t, u)
    y = u + t^2 + 1;
end
	\end{Verbatim}
\end{matlablisting}
Начальное условие: $u_0 = 0.5$

b)

Python:
\begin{minted}[linenos=false,frame=lines,framesep=2mm,baselinestretch=1.2,fontsize=\footnotesize]{python}
import numpy as np

def f(t, u):
    om = np.array([np.sin(t), np.cos(t), np.sin(t + np.pi/4)])
    Omega = np.array([[0, -om[2], om[1]],
                      [om[2], 0, -om[0]],
                      [-om[1], om[0], 0]])
    return np.dot(Omega, u)
\end{minted}

MATLAB:
\begin{matlablisting}
	\begin{verbatim}
function y = f(t, u)
    om    = [ sin(t) cos(t) sin(t+pi/4) ];
    Omega = [ 0     -om(3)  om(2); 
              om(3)  0     -om(1);
             -om(2)  om(1)  0           ];
    y = Omega * u;
end
	\end{verbatim}
\end{matlablisting}

Начальное условие: $u_0 = [1; -0.5; 0.6];$\\
Временной отрезок для обеих функций - от 0 до 1.\\
Провести 7 расчетов на сгущающихся вдвое сетках, начиная с минимально возможной сетки из 1 интервала.\\
Для первой функции построить график эффективного порядка метода от числа интервалов сетки (по последнему узлу, т.е. в последнем узле сетки при $t=1$), для второй - построить график решения (3 кривые на одном графике). 
\item Реализовать явную схему Рунге-Кутты в общем виде. Для отладки использовать 7-стадийную схему Хаммуда 6 порядка:  

Python:
\begin{minted}[linenos=false,frame=lines,framesep=2mm,baselinestretch=1.2,fontsize=\footnotesize]{python}
import numpy as np

butcher = np.array([
    [0, 0, 0, 0, 0, 0, 0],
    [4/7, 0, 0, 0, 0, 0, 0],
    [115/112, -5/16, 0, 0, 0, 0, 0],
    [589/630, 5/18, -16/45, 0, 0, 0, 0],
    [229/1200-29/6000*5**0.5, 119/240 - 187/1200*5**0.5, -14/75+34/375*5**0.5, -3/100*5**0.5,
    0, 0, 0],
    [71/2400 - 587/12000*5**0.5, 187/480 - 391/2400*5**0.5, -38/75 + 26/375*5**0.5,
    27/80 - 3/400*5**0.5, (1+5**0.5)/4, 0, 0],
    [-49/480+43/160*5**0.5, -425/96+51/32*5**0.5, 52/15-4/5*5**0.5,
    -27/16+3/16*5**0.5, 5/4-3/4*5**0.5,5/2-0.5*5**0.5, 0]
])

a = np.array([0, 4/7, 5/7, 6/7, (5-5**0.5)/10, (5+5**0.5)/10, 1])
b = np.array([1/12, 0, 0, 0, 5/12, 5/12, 1/12])

\end{minted}
\newpage
MATLAB:
\begin{matlablisting}
	\begin{verbatim}
butcher = [ 0                      0                     ...
            0                      0                     ...
            0                      0                     0;
            4/7                    0                     ...
            0                      0                     ... 
            0                      0                     0;
            115/112                -5/16                 ...  
            0                      0                     ...
            0                      0                     0;
            589/630                5/18                  ...
           -16/45                  0                     ...
            0                      0                     0;
            229/1200-29/6000*5^.5  119/240-187/1200*5^.5 ...
           -14/75+34/375*5^.5      -3/100*5^.5           ...        
            0                      0                     0;
            71/2400-587/12000*5^.5 187/480-391/2400*5^.5 ...
           -38/75+26/375*5^.5      27/80-3/400*5^.5      ...
            (1+5^.5)/4             0                     0;
           -49/480+43/160*5^.5     -425/96+51/32*5^.5    ...   
            52/15-4/5*5^.5         -27/16+3/16*5^.5      ...
            5/4-3/4*5^.5           5/2-0.5*5^.5          0 ];
a       = [ 0    4/7 5/7 6/7 (5-5^.5)/10 (5+5^.5)/10 1     ];
b       = [ 1/12 0   0   0   5/12        5/12        1/12  ];
	\end{verbatim}
\end{matlablisting}

Провести 7 расчетов на сгущающихся вдвое сетках, начиная с минимально возможной сетки из 1 интервала.\\
Протестировать на тех же тестовых функциях, построить такие же графики.
\end{enumerate}
