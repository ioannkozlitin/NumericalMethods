\documentclass{article}
\usepackage[english, russian]{babel}
\usepackage{listings}
\usepackage{amsmath}
\usepackage{minted}

\title{Теоретический материал к семинару №6}
\date{}
\begin{document}
\maketitle

Видоизменим краевую задачу из предыдущего семинара, сделав из нее задачу на собственные значения
\begin{equation} \label{c6eq1}
	\begin{cases}
		 \displaystyle \frac{d}{dx} \left( \left( k_0 + k_1 u^2 \right) \frac{du}{dx} \right) + \lambda u = 0 \\
		u(a) = u(b) = 0
	\end{cases},
\end{equation}
где $[a; b]$ - отрезок, на котором ищется решение.

Разностная схема легко получается из схемы для краевой задачи заменой $-h^2 f_n$ на $h^2 \lambda u_n$:
\begin{equation} \label{c6eq2}
	\begin{cases}
		\displaystyle \left( u_{n+1} - u_n \right) \left( k_0 + k_1 \frac{u_n^2 + u_{n+1}^2}{2} \right) - \left( u_n - u_{n-1} \right) \left( k_0 + k_1 \frac{u_n^2 + u_{n-1}^2}{2} \right) + h^2 \lambda u_n = 0 \\
		u_0 = u_N = 0
	\end{cases}.
\end{equation}
Здесь $h$ - шаг равномерной сетки, $N$ - число интервалов сетки.

Поскольку это нелинейная задача на собственные значения, то для ее решения надо использовать метод дополненного вектора – вариант метода Ньютона для задач на собственные значения.

В этом методе строится новый вектор   по следующему правилу
\begin{equation} \label{c6eq3}
	\begin{cases}
	\nu_n = u_n, \qquad 0 \leq n \leq N\\
	\nu_{N+1} = \lambda
	\end{cases}.
\end{equation}
Добавление новой переменной потребует увеличения на одно числа уравнений – иначе задача не будет иметь однозначного решения. Так мы снова приходим к необходимости постановки дополнительного граничного условия в задаче на собственные значения.

Дальше задача решается методом Ньютона, подобно тому, как это делалось при решении нелинейной краевой задачи в предыдущем семинаре. Переделки программы будут минимальными.

В данном случае очень полезно провести серию расчетов на сгущающихся сетках и понаблюдать за сходимостью серии найденных собственных значений к некоторому пределу. Это важно, чтобы вовремя заметить «перескок» на другое собственное значение, если он будет иметь место. При «перескоке» серию расчетов придется повторить с другим начальным приближением.

Ясно, что результат расчета на более грубой сетке следует использовать в качестве начального приближения при расчете на более подробной. Для равномерной сетки и сгущения сетки в 2 раза значения в нечетных узлах переносятся непосредственно (1 в 3, 2 в 5 и т.п.), а в четных получаются интерполяций – полусуммой соседних нечетных узлов (например $u_2 = 0.5(u_1+u_3)$). Вычисленное на грубой сетке собственное значение переносится на подробную непосредственно, выполняя роль начального приближения.

К полученному в результате серии расчетов набору приближенных значений $\lambda$ можно применять методы апостериорной оценки погрешности решения Ричардсона и Эйткена. Используя рекуррентное сгущение, можно получить результат с высокой точностью даже на не слишком подробных сетках. 

\section*{Задачи к семинару №6}
Задачу следует решать на отрезке $[0; 1]$, $k_0 = 1$, $k_1 = 0.5$, начальное приближение для собственного значения $\lambda_0 = 40$, самая первая сетка пусть имеет 8 интервалов, последняя – 512  (серия из 7 расчетов). 

Дополнительное граничное условие выглядит следующим образом (здесь $\texttt{\lstinline|u|}$ - дополненный вектор):

Python:
\begin{minted}[linenos=False,frame=lines,framesep=2mm,baselinestretch=1.2,fontsize=\footnotesize]{python}
u[(len(u)//2) + 1] - u[(len(u)//2) - 1] - 2 * h = 0.
\end{minted}
MATLAB:
\begin{matlablisting}
	\begin{verbatim}
u(end/2+1) - u(end/2-1) - 2*h = 0.
	\end{verbatim}
\end{matlablisting}
В данном случае ставится условие на производную собственной функции в середине отрезка (она должна быть равна 1).

Построить график $\lambda$ от номера расчета. Также вывести на график собственную функцию с самой подробной сетки. По полученному набору $\lambda$ определить эффективный порядок метода и получить апостериорную оценку погрешности. С помощью техники рекуррентных сгущений вычислить собственное значение с максимально возможной точностью.

\end{document}