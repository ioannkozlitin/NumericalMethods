\section{Задание к семинару №9}
Решить однородное уравнение теплопроводности с граничными условиями Дирихле 
\begin{equation} \label{c9eq1}
	\begin{cases}
		\displaystyle \frac{\partial u}{\partial t} = k \frac{\partial^2 u}{\partial x^2} \\
		\displaystyle u(x,0) = e^{-(x-5)^4} + 0.01x \\
		u(0,t) = u(0,0) \\  
		u(a,t) = u(a,0)
	\end{cases}
\end{equation}
на отрезке $x \in [0; a]$ при $t \in [0; T]$ Выбрать $a = 20$ и $T = 20$, а коэффициент теплопроводности $k = 2$. Шаг по пространству $h = 0.01$, шаг по времени $\tau = 0.05$. Расчет проводить с помощью комплексной схемы Розенброка. После каждого временного слоя выводить решение на текущем слое для получения анимационной картинки. Чтобы избежать постоянного изменения масштаба графика рекомендуется после команды $\texttt{\lstinline|plot|}$ вставить команду $\texttt{\lstinline|axis([0 a 0 1])|}$, не забыв далее поставить команду $\texttt{\lstinline|pause(1e-6)|}$.
Тот же расчет необходимо повторить с граничными условиями Неймана
\begin{equation} \label{c9eq2}
	u_x(0,t) = u_x(a,t) = 0.
\end{equation}

\subsection{Указания}
\begin{enumerate}
\item Счет будет идти быстрее, если использовать аппарат разреженных матриц MATLAB (смотри документацию к функции $\texttt{\lstinline|spdiags|}$).
\item Граничные условия надо включить в оператор пространственного дифференцирования $\Lambda_x$, видоизменив в нем первую и последнюю строки. Поскольку в данной задаче коэффициент теплопроводности постоянен, то оператор $\Lambda_x$ можно вычислить один раз до начала расчета.
\item Для аппроксимации граничных условий Неймана со 2 порядком следует использовать метод фиктивных точек.
\end{enumerate}


