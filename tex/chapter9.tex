\documentclass{report}
\usepackage{amsmath}
\usepackage{amsfonts}
\usepackage{graphicx}
\usepackage[T2A]{fontenc}      
\usepackage[russian]{babel}
\usepackage{listings} 
\usepackage{titlesec}

\titleformat{\section}[block]{\filcenter\bfseries\large}{\thesection}{1em}{}
\titleformat{\subsection}[block]{\filcenter\bfseries\normalsize}{\thesubsection}{1em}{}

\begin{document}


\section*{Задание к семинару №9}
Решить однородное уравнение теплопроводности с граничными условиями Дирихле 
\begin{equation} \label{c9eq1}
	\begin{cases}
		\displaystyle \frac{\partial u}{\partial t} = k \frac{\partial^2 u}{\partial x^2} \\
		\displaystyle u(x,0) = e^{-(x-5)^4} + 0.01x \\
		u(0,t) = u(0,0) \\  
		u(a,t) = u(a,0)
	\end{cases}
\end{equation}
на отрезке $x \in [0; a]$ при $t \in [0; T]$. Выбрать $a = 20$ и $T = 20$, а коэффициент теплопроводности $k = 2$. Шаг по пространству $h = 0.01$, шаг по времени $\tau = 0.05$. Расчет проводить с помощью комплексной схемы Розенброка. После каждого временного слоя выводить решение на текущем слое для получения анимационной картинки. Чтобы избежать постоянного изменения масштаба графика, рекомендуется после команды \texttt{\lstinline|plot|} вставить команду \texttt{\lstinline|axis([0 a 0 1])|}, не забыв далее поставить команду \texttt{\lstinline|pause(1e-6)|}.
Тот же расчет необходимо повторить с граничными условиями Неймана
\begin{equation} \label{c9eq2}
	u_x(0,t) = u_x(a,t) = 0.
\end{equation}

\section*{Приложение}
\begin{enumerate}
\item Счет будет идти быстрее, если использовать библиотеку SciPy для работы с разреженными матрицами. Например, функция \texttt{\lstinline|scipy.sparse.diags|} позволяет создать разреженную матрицу с диагоналями, что значительно ускорит вычисления.
\item Граничные условия нужно включить в оператор пространственного дифференцирования $\Lambda_x$, видоизменив в нем первую и последнюю строки. Поскольку в данной задаче коэффициент теплопроводности постоянен, оператор $\Lambda_x$ можно вычислить один раз до начала расчета.
\item Для аппроксимации граничных условий Неймана со 2 порядком следует использовать метод фиктивных точек.
\end{enumerate}

Пример использования разреженных матриц в Python с использованием SciPy:

\begin{lstlisting}[language=Python]
impimport numpy as np
import scipy.sparse as sp

N = 5

main_diag = 3 * np.ones(N)

off_diag = -1 * np.ones(N-1)

A = sp.diags([main_diag, off_diag, off_diag], [0, -1, 1], format='csr')

\end{lstlisting}

В этом примере: 
\begin{itemize}
    \item \texttt{main\_diag}: массив, содержащий значения на главной диагонали матрицы.
    \item \texttt{off\_diag}: массив, содержащий значения на диагоналях ниже и выше главной диагонали.
    \item \texttt{sp.diags([main\_diag, off\_diag, off\_diag], [0, -1, 1], format='csr')}:
    \begin{itemize}
        \item Первый аргумент: список массивов, каждый из которых содержит значения для соответствующей диагонали.
        \item Второй аргумент: список целых чисел, указывающих позиции диагоналей относительно главной диагонали. 0 соответствует главной диагонали, -1 — диагонали ниже главной, 1 — диагонали выше главной.
        \item \texttt{format='csr'}: указывает формат хранения разреженной матрицы (в данном случае, CSR — Compressed Sparse Row).
    \end{itemize}
\end{itemize}

\end{document}
